\documentclass[a4paper]{book}

\usepackage[utf8]{inputenc}
\usepackage{ngerman}
\usepackage[a4paper]{geometry}
\usepackage{parskip}
\usepackage{graphicx}
\usepackage{amsmath}
\usepackage{amssymb}
\usepackage{amsthm}
\usepackage{booktabs}
\usepackage[space]{grffile}
\usepackage[dvipsnames]{xcolor}
\usepackage{verbatim}

\newcommand*{\R}{\mathbb{R}}
\newcommand*{\N}{\mathbb{N}}
\newcommand*{\Z}{\mathbb{Z}}
\newcommand*{\Q}{\mathbb{Q}}
\newcommand*{\C}{\mathbb{C}}

\newcommand*{\sgn}{\operatorname{sgn}}

\theoremstyle{plain}
\newtheorem{theorem}{Satz}[chapter]
\newtheorem{definition}{Definition}[chapter]
\newtheorem{solution}{Lösung}[chapter]
\newtheorem{exercise}{Übung}[chapter]
\newtheorem{solutionprocess}{Lösungsweg}[chapter]
\newtheorem{proofsummary}{Beweiszusammenfassung}[chapter]
\newtheorem{alternativeproof}{Alternativer Beweis}[chapter]

\theoremstyle{remark}
\newtheorem{hint}{Hinweis}[chapter]
\newtheorem{warning}{Warnung}[chapter]
\newtheorem{example}{Beispiel}[chapter]
